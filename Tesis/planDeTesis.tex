\documentclass{article}

\usepackage[utf8]{inputenc}
\usepackage[T1]{fontenc}
\usepackage[spanish]{babel}
\usepackage{microtype}
\usepackage[numbers]{natbib}
\usepackage[hidelinks=true]{hyperref}

\title{Plan de Tesis\\\vspace{1em}\Large Inference of Socioeconomic Status\\in a Communications Graph:\\A Bayesian Approach}
\author{{\Large Martín Fixman}\\Director: Carlos Sarraute\\Co-director: Esteban Feuerstein}
\date{{\large Facultad de Ciencias Exactas y Naturales}\\{\large Universidad de Buenos Aires}\\Buenos Aires, 2017}

\bibliographystyle{ieeetr}

\begin{document}

\maketitle

\section*{}

Obtener y procesar datos demográficos y sociológicos fueron uno de los procesos más importantes para entender fenómenos que afectan a toda la población desde por lo menos el Siglo XVII, y encontrar formas simples e intuitivas de visualizarlos tiene un gran impacto en nuestra manera de entender los datos~\citep{minard1844,snow1855}. Formas comunes de obtener datos cuantitativos de estratificación económica usualmente involucran investigación de archivos o encuestas sociales~\citep{bulmer1977}; sin embargo esos métodos no pueden presentar datos que sean simultáneamente exactos, actualizados, y que aplican a una población grande sin depender de métodos estadísticos.

El autor obtuvo acceso a metadatos de conversaciones telefónicas de los usuarios de cierta compañía de telecomunicaciones, con las que es posible definir relaciones entre ellos. A la vez se obtuvieron datos sobre los usuarios de un banco de manera de que sea posible hacer ``merge'' entre estas dos bases de datos. De esta manera, es posible conocer los datos de las cuentas bancarias de los usuarios de la telco además de sus interrelaciones.

A pesar de ser un área de trabajo relativamente nueva, la inferencia de datos de usuario por telecomunicaciones tiene una gran cantidad de publicaciones relevantes. En un trabajo importante en este área, Óskardottir et~al.\ demuestran varios posibles métodos de inferencia de churn (usuarios que pueden abandonar la compañía) en una red de telecomunicaciones~\cite{oskardottir2016}. En esta se demuestra que no hay una sola manera obvia de crear una inferencia para el tipo de inferencias que usualmente son útiles, sino que hay muchos posibles métodos que pueden servir para estos casos sin ninguna manera de saber de antemano cuál es el mejor.

También hay una cantidad significativa de investigación existente usando datos bancarios. Leo et~al.\ usan datos similares para inferir varias correlaciones de alto orden entre tipos de compras y usuarios~\cite{leo2015socioeconomic}. Adicionalmente, el autor usa el mismo set de datos para investigar correlaciones de patrones de consumición en redes socioeconómicas~\cite{leo16correlations}.

Un posible trabajo que hace uso de estos datos es la \emph{Inferencia de Nivel Socioeconómico} de los usuarios. Para esto, una posibilidad es usar algún método general de \emph{Machine Learning} sobre datos acumulados de los usuarios como los descritos en~\cite{oskardottir2016}. Sin embargo, también es posible calcular y aprovechar el nivel de \emph{Homofilia Social} explicado por Mcpherson et~al.~\cite{mcpherson2001birds} y, usando conocimientos en \emph{Inferencia Bayesiana}, encontrar un mejor algoritmo para este tipo de inferencias.

Una parte de este trabajo ya fue publicada por el autor y presentada en la \emph{Conferencia Internacional en Avances en Análisis y Minería de Redes Sociales} (ASONAM) en el año 2016~\cite{fixman2016bayesian}. En esta conferencia se pudo observar lo novedoso e interesante del problema al público científico general.

\bibliography{bibliography/sna}{}

\end{document}
