% !TEX root = tesis.tex

\chapter{Introduction}

% MOTIVATION
\section{Motivation of the Thesis}

In recent years, we have witnessed an exponential growth in the capacity to gather, store and manipulate massive amounts of data across a broad spectrum of disciplines: in astrophysics our capacity to gather and analyze massive datasets from astronomical observations has significantly transformed our capacity to model the dynamics of our cosmos; in sociology our capacity to track and study traits from individuals within a population of millions is allowing us to create social models at multiple scales, tracking individual and collective behavior both in space and time, with a granularity not even imagined twenty years ago.

In particular, mobile phone datasets provide a very rich view into the social interactions and the physical movements of large segments of a population. The voice calls and text messages exchanged between people, together with the call locations (recorded through cell tower usages), allow us to construct a rich social graph which can give us interesting insights on the users' social fabric, detailing not only particular social relationships and traits, but also regular patterns of behavior both in space and time, such as their daily and weekly mobility patterns~\cite{gonzalez2008understanding,ponieman2013human,sarraute2015city}.

Demographic factors play an important role in the constitution and preservation of social links. In particular concerning their age, individuals have a tendency to
establish links with others of similar age. This phenomenon is called age homophily~\cite{mcpherson2001birds}, and has been verified in mobile phone communications graph~\cite{blumenstock2010mobile,sarraute2014} as well as the Facebook graph~\cite{ugander2011anatomy}.


% PREVIOUS WORK ON THIS PROBLEM

Economic factors are also believed to have a determining role in both the social network's structure and dynamics. However, there are still very few large-scale quantitative analyses on the interplay between economic status of individuals and their social network. In~\cite{leo2015socioeconomic}, the authors analyze the correlations between mobile phone data and banking transaction information, revealing the existence of social stratification. They also show the presence of socioeconomic homophily among the networks participants using users' income, purchasing power and debt as indicators.
The authors of \cite{Luo2017inferring} studied the correlation between the position of a node in a mobile phone communications graph and its socio-economic status. They showed that the position and topological attributes in the graph can be used to generate inferences of the users' financial status.
In particular the study \cite{Luo2017inferring} shows the value of the Collective Influence~\cite{morone2015influence} as a topological attribute for the prediction of individual financial status.


% SUMMARY OF THE NEW APPROACH
\section{Summary of our Approach}

In this work, we leverage the socioeconomic homophily present in the cellular phone network to generate inferences of socioeconomic status in the communication graph. To this aim we will use the following data sources: (i) the Call Detail Records (CDRs) from the operator allow us to construct a social graph and to establish social affinities among users; (ii) banking reported income for a subset of their clients obtained from a large bank data source. We then construct an inferential algorithm that allows us to predict the socioeconomic status of users close to those for which we have banking information. To our knowledge, this is the first time both mobile phone and banking information has been integrated in this way to make inferences based on a social telecommunication graph.
Part of this work was published in~\cite{Fixman2016bayesian}.

The work done on this thesis is based the hypotheses that there is a significant level of homophily between a person's socioeconomic level and the one from its contacts (\cref{subsec:income_homophily}), and that using this correlation we can infer the first from the second (\cref{subsec:prediction_algorithm}).
At the same time, this thesis presents several ``conventional'' Machine Learning algorithms (\cref{sec:comparison}) along with an inference algorithm based in Bayesian Inference which works thanks to the correlation hypothesis (\cref{sec:inference_methodology}). This extra information should give this algorithm better results than the conventional ones.

Multiple strategies can be used to generate network features based on the CDRs. For instance, in \cite{oskarsdottir2016} the authors evaluate different collective inference methods applied to the churn prediction problem. Furthermore, the work \cite{oskarsdottir2017social} studies the impact of the social graph definition on the performance of the prediction methods. This motives the second part of the thesis, where we perform a comparative study of methods to generate network features for the nodes in the communication graph, and evaluate their impact on the inference of the income. We also compare the effectiveness of machine learning methods such as Logistic Regression and Random Forest on the different feature sets.

\section{Summary of Results}

The final results are presented in \cref{sec:results,sec:comparison}. This section presents a short summary.
\begin{itemize}
	\item When using the Bayesian Algorithm, the \emph{Area Under the Curve} (which is the target metric user in this thesis) is maximized when the socioeconomic comparison is done by number of contacts (\cref{tab:bayesresults}) to a value of \num{0.746}.
	\item Of the common machine learning methods used, the ones which use the labels of the neighbouring users to make a prediction have a better result of the ones which don't. However, even in this case the results are worse than in the Bayesian Algorithm (\cref{sec:comparison_results}).
\end{itemize}

\section{Organisation of the Thesis}

The remainder of the thesis is separated into 7 chapters.

\begin{description}
\item[\Cref{chap:theoretical_intro}] provides an introduction to the theoretical ideas used in the thesis: the concept of homophily in social networks, introductions to concepts in Bayesian probability and machine learning, and some techniques used to define the level of homophily in the dataset.
\item[\Cref{chap:related_work}] reviews some of the related work on correlations in social-economic networks and on relation between socioeconomic status and mobile phone use that was used as a base for this thesis.
\item[\Cref{sec:dataset}] reviews the telecommunications and bank sources used in this study, how they work together, and also some extra insights about the data that can be found after merging both datasets.
\item[\Cref{sec:inference_methodology}] presents the \emph{Bayesian Algorithm}, used as the main inference algorithm in this thesis. In the first part, it presents a theoretical justification of its correctness using the dataset. Later, it formally presents the algorithm and its possible variations.
\item[\Cref{sec:results}] contains a high level description of the testing environment and of the evaluation method of the \emph{Bayesian Algorithm}. It later finds optimal hyperparameters and, in \cref{tab:bayesresults}, presents the final results of the algorithm.
\item[\Cref{sec:comparison}] Presents other algorithms of differing complexity that work in the same dataset as the previous one, including ones based in common \emph{Machine Learning} with novel feature extraction methods. The final results are presented in \cref{sec:comparison_results}.
\item[\Cref{chap:conclusions}] Presents the conclusions of the work, along with some possible work to be done in the future using this same dataset.
\end{description}
