\chapter{Conclusions}

\section{Main Findings}

This thesis is based on the combination of two separate but related data sources, the multiset of \emph{Mobile Phone} CDRs $P$ and $S$, and the \emph{Bank Information} $B$ for another set of users, both of which have some intersection.
With this data we created a method to predict the \emph{Socioeconomic Index} of the users with no income information based on the demonstrated significant amount of homophily between the income of participants of a call.

This presents a model for predicting the \emph{Socioeconomic Index} of users of the telco, the \emph{Bayesian Model}, which when using the optimal hyperparameters specified in \cref{tab:bayesresults} $\{\varpi = \contacts; \Theta = 0.394; \tau = 0.514\}$ has an \emph{Area Under the Curve} of \num{0.746}, an F\textsubscript{1}-measure of {0.723}, and an F\textsubscript{4}-measure of {0.783}.

In order to do this we first separate the dataset into two similarly sized sets of \emph{Low Income} and \emph{High Income} users. For each user whose \emph{Socioeconomic Level} should be predicted (or for users which are part of the \emph{Testing Set} when choosing the hyperparameters and testing the performance of the method), a \emph{Beta Distribution} is generated whose parameters are the amount of contacts of each category.
These distributions are later used to predict the \emph{Socioeconomic Index} of each user by finding whether the \emph{Quantile Function} of this distribution at some value is greater than some threshold.

To validate this method, many metrics were compared with both basic classifiers and with \emph{Machine Learning} methods using complex feature extraction methods.
While some of the latter gave good results for these methods, the \emph{Bayesian Method} was clearly the best one between all of those.

One of the interesting parts of this analysis is that, while the \emph{Machine Learning} methods usually improve when more data is given in the input, the \emph{Bayesian Method} has better results while still using a much smaller input.
This way, this thesis' results are consistent with the research made by Lu~et.~al~\cite{lu2003link}, which also uses a \emph{Graph Model} as the input of a prediction algorithm and finds that the best results are obtained with a simple feature extraction method, and with a similar work by Óskardottir~et.~al~\cite{oskarsdottir2016} where, in addition to the same conclusion as in the previous article, a complex model of \emph{Iterative Classification} results in a worse prediction.

Our proposed inference methodology is useful in concrete applications, not only as an estimate of socioeconomic attributes of users lacking banking history, but also could be used as a predictor of many other properties of the users which are part of any network.

\section{Future Work}

Instead of staying with two income categories, the model can be extended to a \emph{Multiple Category Model} by finding, for each user, the probability that it belongs to each model.
Instead of defining a \emph{Beta Distribution} for each user, it would be possible to find a \emph{Dirichlet Distribution} where for each of the result distributions a model such as the one presented in \cref{subsec:modelling_users} is defined.

Using this same dataset it would also be possible to predict other properties of each user of the telco, such as \emph{Age}, \emph{Gender}, or \emph{Defaulter Status}. Indeed, many of these predictions won't need a model that's too different to the one presented in this thesis, having those predictions has many practical results.
