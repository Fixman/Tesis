% !TEX root = tesis.tex

\begin{abstract}

Obtaining and processing demographical and sociological data have been some of the most important processes for understanding population-wide phenomena since at least 17th century~\cite{friendly2006}, and finding simple and intuitive ways of visualizing them has a big impact in our ways of understanding the data~\cite{minard1844,snow1855}. Common ways of obtaining useful qualitative data on socioeconomic stratification usually involved archival research or social surveys~\cite{bulmer1977},
% ; however those methods can't present data that's at the same time exact, up to date, and that applies to a large population without having to rely on statistical methods.
and rely on statistical methods.

Telecommunication operators (``telcos'') have access to a wealth of information about their users' communications and habits~\cite{huurdeman2003}, but the ability to store and process that data has taken large strides in the last few years thanks to new and more powerful computers and data mining techniques. The same can be said for sociological and economic information owned by banks and credit cards, and the relation between these two data sources.

Large scale data mining of data from the telecommunications industry is a relatively new area that's been so far mostly used for internal applications~\cite{han2002emerging}, but the gigantic wealth of real-time sociological data has been of interest for academic purposes related to sociology. This thesis builds on methods used by Óskarsdottir et.\ al.~\cite{oskarsdottir2016} and Singh et.\ al~\cite{singh2013predicting}, along with a large dataset of information for a certain telco and a large bank to find that the income distribution of the users follows closely (but not exactly) the income distribution of the whole population.

We have observed a strong homophily between the incomes of contacts in the telco, which along with the uneven distribution of wealth in the population is leveraged to create a methodology, grounded in Bayesian statistics, to infer socioeconomic level of a large subset of users in the network without banking information which is very accurate at $\AUC = 0.746$. The Bayesian method is later compared to several other methods based on supervised machine learning to prove that, even though it uses less input information, it's a better predictor of social features in this particular kind of network.

% In this work, we examine the socio-economic correlations present among users in a mobile phone network. First, we find that the distribution of income for a subset of users, for which we have income information given by a large bank, follows closely, but not exactly, the income distribution for the whole population. We also show the existence of a strong socio-economic homophily in the mobile phone network, where users linked in the network are more likely to have similar income. The main contribution of this work is that we leverage this homophily in order to propose a methodology, based on Bayesian statistics, to infer the socio-economic status for a large subset of users in the network (for which we have no banking information). With our proposed algorithm, we create an estimator for two categories (low and high income) which significantly outperforms both a simpler method based on a frequentist approach and several common \emph{Machine Learning} methods.

\end{abstract}
