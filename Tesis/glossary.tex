\chapter*{Symbol Glossary}

This thesis uses a large amount of symbols for different properties and data. To simplify its comprehension, the following table contains information about some of the most commonly used ones.

\begin{table}[h]
\centering
\begin{tabularx}{\textwidth}{>{\large\bfseries}l X >{\hspace{1em}} l}

\toprule
\ct{Symbol} & \ct{Definition} & \ct{Section} \\

\midrule
	$P$ & Multiset of all voice calls in the graph. & \ref{subsec:dataset_description} \\
	$S$ & Multiset of all SMS in the graph. & \ref{subsec:dataset_description} \\
	$V$ & Set of users in the telco network. & \ref{subsec:dataset_description} \\
	$E$ & Set of edges between users in the telco network, along with their communication properties. & \ref{subsec:dataset_description} \\
	$G$ & Social graph: a tuple containing both $V$ and $E$. & \ref{subsec:dataset_description} \\
	$B$ & Set with banking data for all users. & \ref{subsec:bank_source} \\

\midrule
	$H_{\{1, 2\}}$ & Set of lower and upper income users, respectively. & \ref{subsec:discrimination_by_wealth} \\
	$\varpi$ & Selected property of users for the Bayesian distribution. & \ref{subsec:modelling_users_frequentist} \\
	$\Betasim_v$ & \emph{Beta distribution} defined for a user $v$ to predict its wealth. & \ref{subsec:modelling_users} \\
	$\Theta$ & Quantile used to define the \emph{Posterior Probability} of a user's category. & \ref{subsec:modelling_users} \\
	$\tau$ & Limit for the \emph{Posterior Probability} of high and low income users. & \ref{subsec:categorizing_users} \\

\midrule
	$B_{\train}$ & \emph{Training Set} used for $B$. & \ref{subsec:train_test_split} \\
	$B_{\test}$ & \emph{Testing Set} used for $B$. & \ref{subsec:train_test_split} \\
	$\hat{B}_{\test}$ & Contacts in the \emph{Testing Set} with some contact with banking data. & \ref{subsec:erasing_uninformative_data} \\
	$\Upsilon$ & Subset of $\hat{B}_{\test}$ with the same amount of low and high income users. & \ref{subsec:rebalancing_labels} \\
	$\Omega$ & Same as $B_{\test}$. & \ref{subsec:inner_outer_graph} \\

\midrule
	$\ego_n$ & Accumulated properties of the user in the \emph{Ego Network of Order $n$} of the users. & \ref{subsec:methods_ml} \\
	$\cat_n$ & Union of the properties of $\ego_n$ with the same properties partitioned by income group. & \ref{subsec:methods_ml} \\

\bottomrule

\end{tabularx}
\end{table}
