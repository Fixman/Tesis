In the following section, we compare the results of the Bayesian method with other classifiers.

\begin{figure}
	\begin{tabularx}{\textwidth}{>{\bfseries}X c c c c c c}
		\toprule
		\textbf{Classifier} & \textbf{Accuracy} & \textbf{Precision} & \textbf{Recall} & \textbf{AUC} & \textbf{$F_1$}-\textbf{score} & \textbf{$F_4$}-\textbf{score} \\
		\midrule
		Método Bayesiano & 0 & 0 & 0 & 0 & 0 & 0 \\
		\midrule
		Random Selection & 0 & 0 & 0 & 0 & 0 & 0 \\
		Majority Voting & 0 & 0 & 0 & 0 & 0 & 0 \\
		\midrule
		Naïve Bayes & 0 & 0 & 0 & 0 & 0 & 0 \\
		K Nearest Neighbors & 0 & 0 & 0 & 0 & 0 & 0 \\
		SVM & 0 & 0 & 0 & 0 & 0 & 0 \\
		Logistic Regression & 0 & 0 & 0 & 0 & 0 & 0 \\
		SGD & 0 & 0 & 0 & 0 & 0 & 0 \\
		\midrule
		Categorized LR & 0 & 0 & 0 & 0 & 0 & 0 \\
		Count Link Model & 0 & 0 & 0 & 0 & 0 & 0 \\
		\bottomrule
	\end{tabularx}
\end{figure}

\subsection{Random Selection}

For completeness sake we created a random classifier for the socioeconomic category of each user.

We applied two other inference methods to the same data and compared their accuracies to our Bayesian model.

\begin{itemize}
	\item \textbf{Random selection} which chooses randomly the category for each user.
	\item \textbf{Majority voting} which decides whether a user is in the high or low income category depending on the category of the majority of its contacts. In case of a tie, the category is chosen randomly.
\end{itemize}

The accuracy of the first method is as expected \num{0.5}, while the accuracy for majority voting is \num{0.66}.
With the Bayesian method we obtain an accuracy of \num{0.71}.
