\chapter{Conclusions}
\label{chap:conclusions}

\section{General Objectives}

This thesis presented several methods of manipulating communication and socioeconomic data to produce useful insights about its users.
The combination of different datasets present an unique opportunity to study group behaviour and to experiment with gathering data that isn't directly present as an input.

The first data source used in this study is the multiset of mobile phone \emph{Call Detail Records} $P$ (for phone calls) and $S$ (for SMS), which is presented in \cref{subsec:dataset_description}.
These datasets contain all the communication done by the users of a certain telco, and having them allowed us to create a \emph{Social Graph} which allows us to find many insights about its members.

The second data source is the set with \emph{Bank Information} about the users $B$, which is presented in \cref{subsec:bank_source}.
While the values in this set aren't perfectly correlated to the socioeconomic index of the users, its data is a good enough proxy for the object of this study.
For this reason, separated the users into 2 distinct groups depending on which side of the median of \$6300\footnotemark{} their monthly salary fell into.

\begin{itemize}
	\item $\mathbf{H_1}$ the set of \emph{Low Income} users, with an income less or equal to \$6300\footnotemark[\value{footnote}] a month.
	\item $\mathbf{H_2}$ the set of \emph{High Income} users, with income greater than \$6300\footnotemark[\value{footnote}] a month.
\end{itemize}

Getting the intersection of both datasets it is possible to create the \emph{Social Graph} $G$, with the same users of the set of \emph{Call Detail Records}, but where a subset of users also contain banking information from the set $B$, as shown in \cref{subsec:banktencomathing}.
This set, after being cleaned of outliers in \cref{subsec:outlier_filtering}, is the main input of the methods used in the thesis.

\footnotetext{The input data, including the source country, is anonymized. The symbol \$ refers to a certain world currency and not necessarily the American Dollar.}

\section{Similar Studies}

This thesis is not the first socioeconomic study done in this dataset.

\Cref{sec:leo_correlations} presents a summary of the work done by Léo, Karsai, et~al.\ to correlate different consumption patterns with the socioeconomic level of these users.
In it, the authors present a model of \emph{Homophily} of calls between users of the same socioeconomic level, and proves that many different properties of the \emph{Social Graph} follow these correlations.

\Cref{sec:luo2017inferring} does a study similar to the previous one, where the author finds a correlation between users in the same socioeconomic level in social network patterns, mainly in the diversity of users in their links.
The study also presents several ways of preventing bias in these comparisons.

\section{The Bayesian Algorithm}

\Cref{sec:inference_methodology}, the most important part of this thesis, presented the general method used for predicting the socioeconomic level of the users given information about the neighbours.
The model uses a Bayesian approach to deal with the uncertainty that comes from not having complete information about the users in the dataset by defining, for every user in the dataset, a \emph{Beta Distribution} with parameters equal to the accumulated information of the edges leaving to high and low income neighbours.

The testing for this method is done in \cref{sec:results}, separating the data into two distinct training and testing sets and ignoring some users in order to get an income distribution that's closer to the one in real life and not being subject to the bias that wealthier users tend to have more contacts.
We refer to this graph as the \emph{Inner Graph}, annotated with the Greek letter $\Upsilon$.

We use two hyperparameters to define which model we use. The first one is $\varpi$, originally presented in \cref{subsec:modelling_users_frequentist}, which shows which property of the users is being accumulated and used as the parameters of the \emph{Beta Distribution}.

\begin{equation}
\varpi \in \left\{ \calls, \etime, \sms, \contacts \right\}
\end{equation}

The second hyperparameter is $\Theta$, presented in \cref{subsec:modelling_users}, which is used for defining which quantile of the cumulative probability function is used to decide the probability of each user to belong to a socioeconomic category.
This value is important since it defines the way in which the model's uncertainty plays a role in the prediction.
Since each user has a certain \emph{Beta Distribution} depending on the properties coming from high and low income users, instead of a single value, the distribution for a user with many neighbours with known socioeconomic index will different from one with fewer, even if they have the same ratio of high to low income contacts.

Following experimentation in \cref{subsec:optimize_theta,subsec:algorithm_performance}, the values for these hyperparameters that maximized \emph{Area Under the Curve} were found.
There are presented in \cref{eq:conclusions_best_values}.

\begin{equation}
\label{eq:conclusions_best_values}
\begin{aligned}
	\varpi &= \contacts \\
	\Theta &= 0.394
\end{aligned}
\end{equation}

This method has a third hyperparameter, $\tau$, whose value doesn't affect the \emph{Area Under the Curve} but defines at which probability of being of high income a user is inferred to be of that category.
This hyperparameter was optimized to maximize \emph{Accuracy}, and in \cref{subsec:contacts_infer} the optimal value, shown in \cref{eq:conclusions_best_value_tau} was found.

\begin{equation}
\label{eq:conclusions_best_value_tau}
	\tau = 0.514
\end{equation}

The resulting metrics for this approach are shown in \cref{tab:bayesian_resulting_metrics}.

\begin{table}
\centering
\begin{tabular}{r r r r r r r}
\toprule

\ct{Accuracy} & \ct{Precision} & \ct{Recall} & \ct{AUC} & \ct{F\textsubscript{1}} & \ct{F\textsubscript{4}} & \ct{time} \\
\midrule

0.693 & 0.665 & 0.792 & 0.746 & 0.723 & 0.783 & \SI{33.155}{\second} \\
\bottomrule

\end{tabular}
\caption{Resulting metrics of applying the \emph{Bayesian Algorithm} to the \emph{Inner Graph}}
\label{tab:bayesian_resulting_metrics}
\end{table}

As shown in \cref{sec:comparison}, these metrics are extremely good, specially for a model that only takes 2 parameters per user and is faster than several more conventional ones.

\section{Comparison with different algorithms}

The result of applying this data to the \emph{Bayesian Algorithm} was compared with different, more conventional methods is presented in \cref{sec:comparison}.

These additional methods can be grouped into 3 groups.

\begin{enumerate}
	\item \textbf{Trivial Algorithms}, used for having a base of comparison.
		\begin{itemize}
			\item Random Selection, which chooses a category randomly.
			\item Majority Voting, which chooses the category for which the majority of the users' neighbours belong.
		\end{itemize}
	\item \textbf{Ego Network} based algorithms, which use accumulated data about the edges of a users' \emph{Ego Network} of a certain level.
		\begin{itemize}
			\item $\ego_{0, \ldots, 2}$, where the subscript denotes the size of the \emph{Ego Network} used.
		\end{itemize}
	\item \textbf{Categorical Ego Network} based algorithms, which besides of using the total values in the previous point it also does separate accumulations of each property of the edges in the \emph{Ego Network} depending on the socioeconomic group of the node at the other end.
		\begin{itemize}
			\item $\cat_{0, \ldots, 2}$, where the subscript denotes the size of the \emph{Ego Network} used.
		\end{itemize}
\end{enumerate}

The algorithms based in the \emph{Ego Network} use many more features per user than the \emph{Bayesian Algorithm}, as shown in \cref{tab:datasettable} and compared to the 2 features user in the latter.
Later, these features are fed to a \emph{Machine Learning} method, either \emph{Random Forest} or a \emph{Logistic Regression}, doing an \emph{Grid Search} with \emph{K-folds} using $K = 5$.

The result of these experiments, and its comparison with the \emph{Bayesian Algorithm}, are shown in \cref{tab:innercomparison}.
There are several conclusions that can be found from these results.

\begin{itemize}
	\item In problems with highly nonlinear input data, like this thesis, methods based in \emph{Random Forest} perform better than methods based in \emph{Logistic Regression}.
	\item Socioeconomic data on the edges, like the amount of wealthy people someone calls, is a much better predictor of socioeconomic level than more general metrics, like the total amount of people called, even when there is less data present of this kind.
	\item Adding more levels to the \emph{Ego Network} of each user that's being used for engineering features adds useful features to the prediction and can make the final metrics better.
	\item However, the precious point also adds an amount of noise. By the time the features contain data about the edges adjacent to the neighbours of the neighbours of the user, the metrics decreased.
	\item While this doesn't always follow, the \emph{Area Under the Curve} is a good measurement of the results of each algorithm, since it mostly raises at the same rate as other metrics.
\end{itemize}

Additionally, a similar experimentation has been done in the \emph{Outer Graph} $\Omega$, whose training and testing data contains all nodes in the graph, including those without any neighbour with known socioeconomic data.
Its results can be seem in \cref{tab:outercomparison}.

These results are understandably worse than in the previous table.
As an interesting result, the lower levels of the \emph{Machine Learning} algorithms now contain less useful information and more noise.
For this reason, the signal-to-noise ratio of the featuresets from a bigger \emph{Ego Network} is higher, and unlike in the previous case the \emph{Area Under the Curve} of both $\cat_2$ and $\ego_2$ are higher than that value for $\cat_1$ and $\ego_1$.

\section{Comparison between the Bayesian Algorithm and the methods based in Machine Learning}

\epigraph{``Plurality is not to be posited without necessity''}{\textit{Occam's Razor}}

The \emph{Bayesian Algorithm} has a better performance than every single algorithm based in common \emph{Machine Learning} algorithms and novel feature extraction methods.
This is remarkable since it uses only 2 features per node, compared to the 48 used by $\cat_1$, the version with the highest resulting metrics, or the rest of the methods whose amount of features are shown in \cref{tab:datasettable}.

A good way to explain this result is using \emph{Occam's Razor} that explains that, among competing hypotheses, the one with the fewest assumptions should be selected.

In the \emph{Machine Learning} methods we use the hypothesis that the features are correlated with the results, and that all of them have a high signal-to-noise ratio.
While this is true in a general sense, as \cref{subsec:income_homophily,fig:homophily_heatmap} show, it is not a perfect correlation and not even using a highly nonlinear method like \emph{Random Forest} could improve on a model with less assumptions.

However, for the \emph{Bayesian Method}, after setting the hyperparameters $\varpi$, $\Theta$, and $\tau$, our only hypotheses is that the more calls, SMS, calling time, or contacts a user has with high income, the higher is the \emph{Certainty} that this user is of this category, and vice-versa.

This is a good general hypothesis, and, as shown in this thesis, strong enough to handle a very good socioeconomic level predictor.
