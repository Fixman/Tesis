\documentclass{article}
\usepackage{pdfcomment}

\begin{document}

\section{Abstract}

La explosi\'on del uso del celular en las comunicaciones en los \'ultimos a\~nos se produjo en un momento donde el poder de procesamiento de datos estaba aumentando de manera contundente. Gracias a estos dos cambios a nivel mundial, podemos usar las comunicaciones como una manera de inferir datos sobre los usuarios.

Modelar los datos de las personas dentro de lo que llamamos \textbf{``Grafo de Comunicaciones''} es crucial para lograr un entendimiento sobre la informaci\'on demogr\'afica de la poblaci\'on. Nosotros proponemos usar datos provenientes de diferentes fuentes para aproximar el nivel socioecon\'omico de cada persona en el grafo.

\section{Introducci\'on}

\pdfcomment{Ac\'a van temas sobre papers referenciados, y un resumen de algunas cosas}

\subsection{Descripcion de datos de entrada}

Los datos usados por este estudio consisten de \textbf{Call Detail Records}, o CDRs, sobre llamados y mensajes de texto provenientes de la red de una compa\~n\'ia telef\'onica Mexicana, en un periodo de \( M \) meses (\( M = 3 \)). Cada CDR contiene los n\'umeros de tel\'efonos anonimizados de los dos participantes de la comunicaci\'on \(\left<o, d\right>\), el tiempo del llamado \(t\), y su duraci\'on \(s\). Un subconjunto de estos llamados tambi\'en contienen la latitud y la longitud de la antena usada para la llamada, \(\left<y, x\right>\).

\end{document}

