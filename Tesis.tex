\documentclass[
10pt,
spanish,
singlespacing, % Single line spacing, alternatives: onehalfspacing or doublespacing
%draft, % Uncomment to enable draft mode (no pictures, no links, overfull hboxes indicated)
%nolistspacing, % If the document is onehalfspacing or doublespacing, uncomment this to set spacing in lists to single
%liststotoc, % Uncomment to add the list of figures/tables/etc to the table of contents
%toctotoc, % Uncomment to add the main table of contents to the table of contents
parskip, % Uncomment to add space between paragraphs
%nohyperref, % Uncomment to not load the hyperref package
headsepline, % Uncomment to get a line under the header
twocolumn
]{article} % The class file specifying the document structure

\usepackage{pdfcomment}
\usepackage{palatino}
\usepackage[utf8]{inputenc} % Required for inputting international characters
\usepackage[T1]{fontenc} % Output font encoding for international characters
\usepackage[margin=2.5cm]{geometry}
\usepackage{changepage}

% \geometry{
% 	paper=a4paper, % Change to letterpaper for US letter
% 	inner=2.5cm, % Inner margin
% 	outer=3.8cm, % Outer margin
% 	bindingoffset=2cm, % Binding offset
% 	top=1.5cm, % Top margin
% 	bottom=1.5cm, % Bottom margin
% 	%showframe,% show how the type block is set on the page
% }


\title{Inferencia del Índice de Sociabilidad}
\author{
	\hspace{-1.725cm} Esteban Feuerstein\\
	\hspace{-1.725cm} Universidad de Buenos Aires\\
	\hspace{-1.725cm} Buenos Aires, Argentcma\\
	\hspace{-1.725cm} \url{efeuerst@dc.uba.ar}
	\and
	Martín Fixman\\
	Universidad de Buenos Aires\\
	Buenos Aires, Argentina\\
	\url{martinfixman@gmail.com}
	\and
	Carlos Sarraute\\
	Grandata Labs\\
	Buenos Aires, Argentina\\
	\url{charles@grandata.com}
}
\date{}

\begin{document}

\maketitle

\section*{Abstract}

La explosi\'on del uso del celular en las comunicaciones en los \'ultimos a\~nos se produjo en un momento donde el poder de procesamiento de datos estaba aumentando de manera contundente. Gracias a estos dos cambios a nivel mundial, podemos usar las comunicaciones como una manera de inferir datos sobre los usuarios.

Modelar los datos de las personas dentro de lo que llamamos \textbf{Grafo de Comunicaciones} es crucial para lograr un entendimiento sobre la informaci\'on demogr\'afica de la poblaci\'on. Nosotros proponemos usar datos provenientes de diferentes fuentes para aproximar el nivel socioecon\'omico de cada persona en el grafo.

\section*{Introducci\'on}

\pdfcomment{Ac\'a van temas sobre papers referenciados, y un resumen de algunas cosas}

\subsection*{Descripci\'on de datos de entrada}

Los datos usados por este estudio consisten de \textbf{Call Detail Records}, o CDRs, sobre llamados y mensajes de texto provenientes de la red de una compa\~n\'ia telef\'onica Mexicana, en un periodo de \( M \) meses (\( M = 3 \)), denotado por \( L \). Cada CDR contiene los n\'umeros de tel\'efonos anonimizados de los dos participantes de la comunicaci\'on \(\left<o, d\right>\), el tiempo del llamado \(t\), y, en el caso de los CDRs de llamados su duraci\'on \(s\). Un subconjunto de estos llamados tambi\'en contienen la latitud y la longitud de la antena usada para la llamada, \(\left<y, x\right>\).

\( L \) contiene llamadas y mensajes cuyo origen o destino son los usuarios de la telco, pero no incluye comunicaciones entre la gran cantidad de usuarios que no son parte de esta. Definiendo \( N \) como los usuarios de la telco y \( L_N \) como las llamadas donde \( o \in N \wedge d \in N \), podemos crear un grafo de comunicaciones \( G_N \) que contenga todas las llamadas de cada uno de sus usuarios.

Adem\'as se usa informaci\'on sobre dep\'ositos, extracciones, y transacciones hechas desde bancarias de uno de los bancos m\'as grandes de M\'exico, algunas de las cuales incluyen el n\'umero de tel\'efono anonimizado de la misma manera que en los CDRs, por lo cual se puede correlacionar con estos. Usamos los datos durante el mismo periodo de \( M \) meses, denotado por \( B \), los cuales incluyen, por cada usuario hasta 4 tel\'efonos \( t_0 \cdots t_3 \), e informaci\'on del monto de sus transacciones \( s_0 \cdots s_n \).

Al igual que en el dataset anterior, \( B \) contiene una gran cantidad de datos sobre usuarios que no pertenecen a la telco. Definimos \( B_N \) como los usuarios del banco donde \( \left( \exists i \right) t_i \in N \), osea que pertenezcan a la telco.

Para este estudio conseguimos informaci\'on demogr\'afica sobre la edad y el g\'enero \( \left<e, g\right> \) de un subconjunto de los nodos \( G_N \) al que denominamos \textbf{Ground Truth}. Estos datos los provee la compan\~n\'ia telef\'onica y no tenemos ning\'un tipo de control sobre como se selecciona.

Para apoyar al estudio, conseguimos los resultados del \'ultimo censo mexicano, encontrado en \pdfcomment{URL del censo ac\'a} y, usando el nivel socioecon\'omico de ciertas \'areas de M\'exico, podemos hacer una primera aproximaci\'on del nivel socioecon\'omico de la poblaci\'on de esas \'areas.

\end{document}

