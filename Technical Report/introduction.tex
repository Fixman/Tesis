\section{Introduction}

In recent years, we have witnessed an exponential growth in the capacity to gather, store and manipulate massive amounts of data across a broad spectrum of disciplines: in astrophysics our capacity to gather and analyse massive datasets from astronomical observations has significantly transformed our capacity to model the dynamics of our cosmos; in sociology our capacity to track and study traits from individuals within a population of millions is allowing us to create social models at multiple scales, tracking individual and collective behavior both in space and time, with a granularity not even imagined twenty years ago.

In particular, mobile phone datasets provide a very rich view into the social interactions and the physical movements of large segments of a population. The voice calls and text messages exchanged between people, together with the call locations (recorded through cell tower usages), allow us to construct a rich social graph which can give us interesting insights on the users' social fabric, detailing not only particular social relationships and traits, but also regular patterns of behavior both in space and time, such as their daily and weekly mobility patterns~\cite{gonzalez2008understanding,ponieman2013human,sarraute2015city}.

Demographic factors play an important role in the constitution and preservation of social links. In particular concerning their age, individuals have a tendency to
establish links with others of similar age. This phenomenon is called  age homophily~\cite{mcpherson2001birds}, and has been verified in mobile phone communications graph~\cite{blumenstock2010mobile,sarraute2014} as well as the Facebook graph~\cite{ugander2011anatomy}.

Economic factors are also believed to have a determining role in both the social network's structure and dynamics. However, there are still very few large-scale quantitative analyses on the interplay between economic status of individuals and their social network. In~\cite{leo2015socioeconomic}, the authors analyze the correlations between mobile phone data and banking transaction information, revealing the existence of social stratification. They also show the presence of socioeconomic homophily among the networks participants using users' income, purchasing power and debt as indicators.

In this work, we leverage the socioeconomic homophily present in the cellular phone network to generate inferences of socioeconomic status in the communication graph. To this aim we will use the following data sources: (i) the Call Detail Records (CDRs) from the operator allow us to construct a social graph and to establish social affinities among users; (ii) banking reported income for a subset of their clients obtained from a large bank data source. We then construct an inferential algorithm that allows us to predict the socioeconomic status of users close to those for which we have banking information. To our knowledge, this is the first time both mobile phone and banking information has been integrated in this way to make inferences based on a social telecommunication graph.
