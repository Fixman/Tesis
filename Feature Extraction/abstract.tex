The patterns of usage of mobile phone communications, as well as the information of the social network graph, allow us to make inferences of users socio-economic attributes, in particular their income level. 
We present here several methods to extract features from mobile phone usage (calls and messages), and compare different combinations of supervised machine learning techniques and sets of features used as input.  
Our experimental results show that the Bayesian inference method based on the communication graph outperforms a test set of machine learning algorithms based on features aggregated by nodes.
 

%Supervised machine learning algorithms that predict features of users based on both existing data of others and information about their social network graph allow us to easily use existing information in existing communications networks to predict unknown data about their users. While these algorithms are still in their infancy, they are still used all over the industry and academia.
%
%This paper presents several feature extraction methods and compares their ability to predict income inference using different metrics.
