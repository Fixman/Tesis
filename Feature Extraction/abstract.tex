The patterns of usage of mobile phone communications, as well as the information of the social network graph, allow us to make inferences of users socio-economic attributes, in particular their income level.
We present here several methods to extract features from mobile phone usage (calls and messages), and compare different combinations of supervised machine learning techniques and sets of features used as input.
Our experimental results show that the Bayesian inference method based on the communication graph outperforms a test set of machine learning algorithms based on features aggregated by nodes.
