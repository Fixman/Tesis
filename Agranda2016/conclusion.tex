\section{Conclusion}

This work is based on the combination of two data sourcse for phone records and banking information. We showed there's a significant level of homphily between the income levels of the participants of a call, and based on this property we presented a Bayesian approach to infer income categories for users in the graph for which we don't have banking data.

We first tried to separate users into 2 categories depending on their income. To do this, we computed the number of calls each user \( u \) makes to member of the same and different categories, and we constructed a Beta distribution for the probability of that user belonging to each category. We later validated this approach by constructing a ROC curve to compare it to random guessing and a simple test to compare it to majority voting. This way we were able to confirm that the method presented in this paper is better than the previous two.

The method proposed can be easily extended to more than 2 categories and the predictor for each category is still better than naïve methods of inference.

Our proposed inference methodology is useful for concrete applications, since it provides an estimation of socio-economic attributes of users lacking banking history, based on their communication network. We also note that this methodology is not restricted to the inference of socio-economic attributes, but is equally applicable to any attribute that exhibits significant homophily in the network.

%\todo{Revisar conclusions}
%
%\todo{Para una proxima version:}
%
%\todo{Ampliar a vecinos de vecinos}
%
%\todo{Comparar con otros metodos}
%
%\todo{Comparar con reacción-difusión (a primeros vecinos)}
%
%\todo{Idea de Alejo: usar los parametros de la Dirichlet (usar un prior no uniforme)}
%
%\todo{Distribucion de income inferido en función de la edad}

