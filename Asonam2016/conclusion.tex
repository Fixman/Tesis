\section{Conclusion}

This work is based on the combined data source 
from mobile phone records and banking information.
We first showed the presence of homophily with respect to the monthly income in the 
communication graph, that is, users of similar income tend to communicate more with each other. Based on the evidenced homophily, we presented a methodology to infer income categories for users in the graph for which we don't have 
banking information, by taking a Bayesian approach. 

To classify users into income categories, we first computed the number of calls a user $u$ makes to members of the different categories, and used them to construct a Beta distribution for the probability of user $u$ to belong to the high income category. We then validated our approach by constructing the ROC curve which clearly shows that our methodology outperforms random guessing.
Finally, we described how we can extend our approach to multiple categories using the Dirichlet distribution.

Our proposed inference methodology is very useful for concrete applications, since it provides an estimation of socio-economic attributes of users lacking banking history, based on their communication network. We also note that this methodology is not restricted to the inference of socio-economic attributes, but is equally applicable to any attribute that exhibits significant homophily in the network.

