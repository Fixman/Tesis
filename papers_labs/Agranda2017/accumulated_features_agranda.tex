% !TEX root = feature_extraction_agranda.tex

\section{Accumulated Graph Features}
\label{sec:accumulatedfeatures}

This section presents several ways of transforming data from the graph $G = \left< V, E \right>$ into individual features for each user $v \in V$.
The aggregations are classified into levels named according to the transformation done to $G$, and they are merged with levels containing less information.
The total amount of columns in each featureset is presented in \cref{tab:features}.

\begin{table}[t]
\centering
\begin{tabular}{>{\bfseries}l r}
\toprule
Level & Features \\
\midrule
$\ego_1$ & \num{8}  \\
$\ego_2$ & \num{16} \\
$\ego_3$ & \num{24} \\
$\cat_1$ & \num{24} \\
$\cat_2$ & \num{48} \\
$\cat_3$ & \num{72} \\
\bottomrule
\end{tabular}
\caption{Amount of total features per level.}
\label{tab:features}
\end{table}

\subsection{User Data --- Level $\ego_1$}
\label{subsec:user_data}

The first accumulated features consist of accumulating the three quantifiable features, \emph{Calls}, \emph{Time} and \emph{SMS}, for every node, separated on whether those features are incoming or outgoing.

\subsection{Higher Order User Data --- Level $\ego_{n > 1}$}
\label{subsec:higherorderuserdata}

The \emph{ego network} of the node $v$ is defined as the graph consisting of $v$ and its neighbors. A simple way to get more features about that node is to accumulate the calls and SMS information about the edges which are \textbf{not} part of the ego network, but contain one endpoint on the border of the ego network.

Similarly, we define the \emph{user data of order $n$}, for any natural number $n$, as the accumulation of calls and SMS information for the nodes which are part of the \emph{ego network of order $n$}, which is the set of nodes which are at distance at most $n$ of $v$, and are not part of the \emph{ego network of order $n - 1$}, which we denote as $\ego_n$.
The level $\ego_1$ contains the information of the regular \emph{user data}, % from \cref{subsec:user_data}, 
while the user data from the \emph{ego network of order $n$} is assigned to $\ego_n$ for $n > 1$.

\subsection{Categorical User Data --- Level $\cat_n$}
\label{subsec:categoricaluserdata}

Another approach to building features is to do an approach similar to the \emph{user data} presented in \cref{subsec:user_data}, but further discriminating each feature which corresponds to each node $v \in V$ and each edge $e \in E$:
when $t \in T$ is the other endpoint of the link $e$, we discriminate whether $t$ corresponds to a user with high or low income.
The resulting new features are of the form represented by the set in \cref{eq:matcatuserdata}. This way we create the datasets $\cat_1 \dots \cat_3$ by using the growing ego networks $\ego_1 \dots \ego_3$.

\begin{equation}
\begin{Bmatrix} in \\ out \end{Bmatrix}
\times
\begin{Bmatrix} calls \\ time \\ sms \\ contacts \end{Bmatrix}
\times
\begin{Bmatrix} low \\ high \end{Bmatrix}
\label{eq:matcatuserdata}
\end{equation}

To prevent overfitting, the set $T$ is partitioned into two disjoint sets, $G$ and $H$, where $G$ contains roughly 75\% of the nodes in $T$ is used to calculate the features, while $H$ contains the other 25\% and is used to train the models.
