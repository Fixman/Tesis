% !TEX root = feature_extraction_agranda.tex

\section{Introduction}

Mobile phone datasets present a rich view into social interactions and physical movements of large segments of the population.
%Thanks to the exponential growth of mobile telecommunications and of the ability to store and process the information generated by mobile phone users, we can generate inferences of economic and social characteristics of users from the telecommunications graph with much higher accuracy than what was possible just a few years ago.
The voice calls and text messages exchanged between people, together with the locations of these communications, allow us to construct a rich social graph which provides insights on the users social fabric.
%, revealing regularities in their behaviours and mobility patterns~\cite{gonzalez2008understanding,ponieman2013human,sarraute2015city}.

There is a strong homophily in the population's communications graph respect to economic variables such as the user's income~\cite{fixmanasonam2016}, which results largely from social stratification between populations of different purchasing power~\cite{leo2015socioeconomic} or purchasing patterns~\cite{Leo2016correlations}.
%, the spending behaviour of each user~\cite{singh2013predicting}, along with other economic indicators.
% This phenomenon is analogous to the homophily and stratification seen respect to other user features in similar social graphs, for instance between age categories~\cite{mcpherson2001birds} in the mobile phone communications graph~\cite{sarraute2014} as well as the Facebook graph~\cite{ugander2011anatomy}.
Additionally, and in part resulting from this stratification, there are different patterns of communication between users of distinct socioeconomic level~\cite{Luo2017inferring}.

Finding the best way to parse Call Detail Records (CDRs) to generate user features and construct their communications graph is still a subject of research.  After describing our data sources, we present several methods of feature extraction from the raw CDR data,
and describe the supervised machine learning algorithms used to predict the \emph{socioeconomic level} of users, given a ground truth for a relatively small subset of the population. In particular we tested the Bayesian approach for income inference presented in~\cite{fixmanasonam2016}.
Finally, we present our experimental results, comparing the performance obtained according to the feature set and the algorithms used.
