\section{Conclusion}

This work is based on the combination of two data sources of mobile phone records and banking information. We showed that there is a significant level of homophily between the income of the participants of a call, and based on this property, we presented a Bayesian approach to infer the income category of users in the graph for which we don't have banking data.

We first classified users into 2 categories depending on their income. To this end, we computed the number of calls each user \( u \) makes to members of the same and different categories, and we constructed a Beta distribution for the probability of user $u$ belonging to each category. We later validated this approach by constructing the ROC curve and computing its accuracy, and compared it to random guessing and to a simpler method based on majority voting. We were able to validate that the method presented outperforms the other two.
Finally, we showed that this approach can be extended to more than 2 categories by using a Dirichlet distribution. 
% The predictor for each category performs better than a frequentist method of inference.

Our proposed inference methodology is useful in concrete applications, since it provides an estimation of socio-economic attributes of users lacking banking history, based on their communication network. We also note that this methodology is not restricted to the inference of socio-economic attributes, but is equally applicable to any attribute that exhibits significant homophily in the network.

%\todo{Revisar conclusions}
%
%\todo{Para una proxima version:}
%
%\todo{Ampliar a vecinos de vecinos}
%
%\todo{Comparar con otros metodos}
%
%\todo{Comparar con reacción-difusión (a primeros vecinos)}
%
%\todo{Idea de Alejo: usar los parametros de la Dirichlet (usar un prior no uniforme)}
%
%\todo{Distribucion de income inferido en función de la edad}

